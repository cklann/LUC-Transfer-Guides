% Generated by Sphinx.
\def\sphinxdocclass{report}
\documentclass[letterpaper,10pt,english]{sphinxmanual}
\usepackage[utf8]{inputenc}
\DeclareUnicodeCharacter{00A0}{\nobreakspace}
\usepackage[T1]{fontenc}
\usepackage{babel}
\usepackage{times}
\usepackage[Bjarne]{fncychap}
\usepackage{longtable}
\usepackage{sphinx}
\usepackage{multirow}


\title{Loyola University Chicago Computer Science - Graduate Handbook Documentation}
\date{March 21, 2013}
\release{1.0}
\author{CS Department}
\newcommand{\sphinxlogo}{}
\renewcommand{\releasename}{Release}
\makeindex

\makeatletter
\def\PYG@reset{\let\PYG@it=\relax \let\PYG@bf=\relax%
    \let\PYG@ul=\relax \let\PYG@tc=\relax%
    \let\PYG@bc=\relax \let\PYG@ff=\relax}
\def\PYG@tok#1{\csname PYG@tok@#1\endcsname}
\def\PYG@toks#1+{\ifx\relax#1\empty\else%
    \PYG@tok{#1}\expandafter\PYG@toks\fi}
\def\PYG@do#1{\PYG@bc{\PYG@tc{\PYG@ul{%
    \PYG@it{\PYG@bf{\PYG@ff{#1}}}}}}}
\def\PYG#1#2{\PYG@reset\PYG@toks#1+\relax+\PYG@do{#2}}

\expandafter\def\csname PYG@tok@gd\endcsname{\def\PYG@tc##1{\textcolor[rgb]{0.63,0.00,0.00}{##1}}}
\expandafter\def\csname PYG@tok@gu\endcsname{\let\PYG@bf=\textbf\def\PYG@tc##1{\textcolor[rgb]{0.50,0.00,0.50}{##1}}}
\expandafter\def\csname PYG@tok@gt\endcsname{\def\PYG@tc##1{\textcolor[rgb]{0.00,0.27,0.87}{##1}}}
\expandafter\def\csname PYG@tok@gs\endcsname{\let\PYG@bf=\textbf}
\expandafter\def\csname PYG@tok@gr\endcsname{\def\PYG@tc##1{\textcolor[rgb]{1.00,0.00,0.00}{##1}}}
\expandafter\def\csname PYG@tok@cm\endcsname{\let\PYG@it=\textit\def\PYG@tc##1{\textcolor[rgb]{0.25,0.50,0.56}{##1}}}
\expandafter\def\csname PYG@tok@vg\endcsname{\def\PYG@tc##1{\textcolor[rgb]{0.73,0.38,0.84}{##1}}}
\expandafter\def\csname PYG@tok@m\endcsname{\def\PYG@tc##1{\textcolor[rgb]{0.13,0.50,0.31}{##1}}}
\expandafter\def\csname PYG@tok@mh\endcsname{\def\PYG@tc##1{\textcolor[rgb]{0.13,0.50,0.31}{##1}}}
\expandafter\def\csname PYG@tok@cs\endcsname{\def\PYG@tc##1{\textcolor[rgb]{0.25,0.50,0.56}{##1}}\def\PYG@bc##1{\setlength{\fboxsep}{0pt}\colorbox[rgb]{1.00,0.94,0.94}{\strut ##1}}}
\expandafter\def\csname PYG@tok@ge\endcsname{\let\PYG@it=\textit}
\expandafter\def\csname PYG@tok@vc\endcsname{\def\PYG@tc##1{\textcolor[rgb]{0.73,0.38,0.84}{##1}}}
\expandafter\def\csname PYG@tok@il\endcsname{\def\PYG@tc##1{\textcolor[rgb]{0.13,0.50,0.31}{##1}}}
\expandafter\def\csname PYG@tok@go\endcsname{\def\PYG@tc##1{\textcolor[rgb]{0.20,0.20,0.20}{##1}}}
\expandafter\def\csname PYG@tok@cp\endcsname{\def\PYG@tc##1{\textcolor[rgb]{0.00,0.44,0.13}{##1}}}
\expandafter\def\csname PYG@tok@gi\endcsname{\def\PYG@tc##1{\textcolor[rgb]{0.00,0.63,0.00}{##1}}}
\expandafter\def\csname PYG@tok@gh\endcsname{\let\PYG@bf=\textbf\def\PYG@tc##1{\textcolor[rgb]{0.00,0.00,0.50}{##1}}}
\expandafter\def\csname PYG@tok@ni\endcsname{\let\PYG@bf=\textbf\def\PYG@tc##1{\textcolor[rgb]{0.84,0.33,0.22}{##1}}}
\expandafter\def\csname PYG@tok@nl\endcsname{\let\PYG@bf=\textbf\def\PYG@tc##1{\textcolor[rgb]{0.00,0.13,0.44}{##1}}}
\expandafter\def\csname PYG@tok@nn\endcsname{\let\PYG@bf=\textbf\def\PYG@tc##1{\textcolor[rgb]{0.05,0.52,0.71}{##1}}}
\expandafter\def\csname PYG@tok@no\endcsname{\def\PYG@tc##1{\textcolor[rgb]{0.38,0.68,0.84}{##1}}}
\expandafter\def\csname PYG@tok@na\endcsname{\def\PYG@tc##1{\textcolor[rgb]{0.25,0.44,0.63}{##1}}}
\expandafter\def\csname PYG@tok@nb\endcsname{\def\PYG@tc##1{\textcolor[rgb]{0.00,0.44,0.13}{##1}}}
\expandafter\def\csname PYG@tok@nc\endcsname{\let\PYG@bf=\textbf\def\PYG@tc##1{\textcolor[rgb]{0.05,0.52,0.71}{##1}}}
\expandafter\def\csname PYG@tok@nd\endcsname{\let\PYG@bf=\textbf\def\PYG@tc##1{\textcolor[rgb]{0.33,0.33,0.33}{##1}}}
\expandafter\def\csname PYG@tok@ne\endcsname{\def\PYG@tc##1{\textcolor[rgb]{0.00,0.44,0.13}{##1}}}
\expandafter\def\csname PYG@tok@nf\endcsname{\def\PYG@tc##1{\textcolor[rgb]{0.02,0.16,0.49}{##1}}}
\expandafter\def\csname PYG@tok@si\endcsname{\let\PYG@it=\textit\def\PYG@tc##1{\textcolor[rgb]{0.44,0.63,0.82}{##1}}}
\expandafter\def\csname PYG@tok@s2\endcsname{\def\PYG@tc##1{\textcolor[rgb]{0.25,0.44,0.63}{##1}}}
\expandafter\def\csname PYG@tok@vi\endcsname{\def\PYG@tc##1{\textcolor[rgb]{0.73,0.38,0.84}{##1}}}
\expandafter\def\csname PYG@tok@nt\endcsname{\let\PYG@bf=\textbf\def\PYG@tc##1{\textcolor[rgb]{0.02,0.16,0.45}{##1}}}
\expandafter\def\csname PYG@tok@nv\endcsname{\def\PYG@tc##1{\textcolor[rgb]{0.73,0.38,0.84}{##1}}}
\expandafter\def\csname PYG@tok@s1\endcsname{\def\PYG@tc##1{\textcolor[rgb]{0.25,0.44,0.63}{##1}}}
\expandafter\def\csname PYG@tok@gp\endcsname{\let\PYG@bf=\textbf\def\PYG@tc##1{\textcolor[rgb]{0.78,0.36,0.04}{##1}}}
\expandafter\def\csname PYG@tok@sh\endcsname{\def\PYG@tc##1{\textcolor[rgb]{0.25,0.44,0.63}{##1}}}
\expandafter\def\csname PYG@tok@ow\endcsname{\let\PYG@bf=\textbf\def\PYG@tc##1{\textcolor[rgb]{0.00,0.44,0.13}{##1}}}
\expandafter\def\csname PYG@tok@sx\endcsname{\def\PYG@tc##1{\textcolor[rgb]{0.78,0.36,0.04}{##1}}}
\expandafter\def\csname PYG@tok@bp\endcsname{\def\PYG@tc##1{\textcolor[rgb]{0.00,0.44,0.13}{##1}}}
\expandafter\def\csname PYG@tok@c1\endcsname{\let\PYG@it=\textit\def\PYG@tc##1{\textcolor[rgb]{0.25,0.50,0.56}{##1}}}
\expandafter\def\csname PYG@tok@kc\endcsname{\let\PYG@bf=\textbf\def\PYG@tc##1{\textcolor[rgb]{0.00,0.44,0.13}{##1}}}
\expandafter\def\csname PYG@tok@c\endcsname{\let\PYG@it=\textit\def\PYG@tc##1{\textcolor[rgb]{0.25,0.50,0.56}{##1}}}
\expandafter\def\csname PYG@tok@mf\endcsname{\def\PYG@tc##1{\textcolor[rgb]{0.13,0.50,0.31}{##1}}}
\expandafter\def\csname PYG@tok@err\endcsname{\def\PYG@bc##1{\setlength{\fboxsep}{0pt}\fcolorbox[rgb]{1.00,0.00,0.00}{1,1,1}{\strut ##1}}}
\expandafter\def\csname PYG@tok@kd\endcsname{\let\PYG@bf=\textbf\def\PYG@tc##1{\textcolor[rgb]{0.00,0.44,0.13}{##1}}}
\expandafter\def\csname PYG@tok@ss\endcsname{\def\PYG@tc##1{\textcolor[rgb]{0.32,0.47,0.09}{##1}}}
\expandafter\def\csname PYG@tok@sr\endcsname{\def\PYG@tc##1{\textcolor[rgb]{0.14,0.33,0.53}{##1}}}
\expandafter\def\csname PYG@tok@mo\endcsname{\def\PYG@tc##1{\textcolor[rgb]{0.13,0.50,0.31}{##1}}}
\expandafter\def\csname PYG@tok@mi\endcsname{\def\PYG@tc##1{\textcolor[rgb]{0.13,0.50,0.31}{##1}}}
\expandafter\def\csname PYG@tok@kn\endcsname{\let\PYG@bf=\textbf\def\PYG@tc##1{\textcolor[rgb]{0.00,0.44,0.13}{##1}}}
\expandafter\def\csname PYG@tok@o\endcsname{\def\PYG@tc##1{\textcolor[rgb]{0.40,0.40,0.40}{##1}}}
\expandafter\def\csname PYG@tok@kr\endcsname{\let\PYG@bf=\textbf\def\PYG@tc##1{\textcolor[rgb]{0.00,0.44,0.13}{##1}}}
\expandafter\def\csname PYG@tok@s\endcsname{\def\PYG@tc##1{\textcolor[rgb]{0.25,0.44,0.63}{##1}}}
\expandafter\def\csname PYG@tok@kp\endcsname{\def\PYG@tc##1{\textcolor[rgb]{0.00,0.44,0.13}{##1}}}
\expandafter\def\csname PYG@tok@w\endcsname{\def\PYG@tc##1{\textcolor[rgb]{0.73,0.73,0.73}{##1}}}
\expandafter\def\csname PYG@tok@kt\endcsname{\def\PYG@tc##1{\textcolor[rgb]{0.56,0.13,0.00}{##1}}}
\expandafter\def\csname PYG@tok@sc\endcsname{\def\PYG@tc##1{\textcolor[rgb]{0.25,0.44,0.63}{##1}}}
\expandafter\def\csname PYG@tok@sb\endcsname{\def\PYG@tc##1{\textcolor[rgb]{0.25,0.44,0.63}{##1}}}
\expandafter\def\csname PYG@tok@k\endcsname{\let\PYG@bf=\textbf\def\PYG@tc##1{\textcolor[rgb]{0.00,0.44,0.13}{##1}}}
\expandafter\def\csname PYG@tok@se\endcsname{\let\PYG@bf=\textbf\def\PYG@tc##1{\textcolor[rgb]{0.25,0.44,0.63}{##1}}}
\expandafter\def\csname PYG@tok@sd\endcsname{\let\PYG@it=\textit\def\PYG@tc##1{\textcolor[rgb]{0.25,0.44,0.63}{##1}}}

\def\PYGZbs{\char`\\}
\def\PYGZus{\char`\_}
\def\PYGZob{\char`\{}
\def\PYGZcb{\char`\}}
\def\PYGZca{\char`\^}
\def\PYGZam{\char`\&}
\def\PYGZlt{\char`\<}
\def\PYGZgt{\char`\>}
\def\PYGZsh{\char`\#}
\def\PYGZpc{\char`\%}
\def\PYGZdl{\char`\$}
\def\PYGZhy{\char`\-}
\def\PYGZsq{\char`\'}
\def\PYGZdq{\char`\"}
\def\PYGZti{\char`\~}
% for compatibility with earlier versions
\def\PYGZat{@}
\def\PYGZlb{[}
\def\PYGZrb{]}
\makeatother

\begin{document}

\maketitle
\tableofcontents
\phantomsection\label{index::doc}


This is a \emph{draft} of the Graduate Handbook for students at Loyola University
Chicago. It is (a) not remotely ready and (b) not at all official.
This is for review purposes only (by the Graduate Program Committee and
the CS Department faculty). We will
announce its availability by late Spring 2013 (around April but possibly
earlier).

Table of Contents


\chapter{Downloading for Offline Reading}
\label{downloading:downloading-for-offline-reading}\label{downloading::doc}\label{downloading:loyola-university-chicago-computer-science-graduate-handbook}

\section{Source}
\label{downloading:source}
The Graduate Handbook is written using the incredible \href{http://sphinx-doc.org/}{Sphinx} documenation tools from the \href{http://python.org}{Python} Community.

You can view the source code at \href{https://bitbucket.org/loyolachicagocs/gradhandbook}{https://bitbucket.org/loyolachicagocs/gradhandbook}.


\section{Other Formats}
\label{downloading:other-formats}
We also provide the following formats for offline reading:
\begin{itemize}
\item {} 
\href{http://gradhandbook.cs.luc.edu/latex/LoyolaComputerScienceGradHandbook.pdf}{PDF} (for printing or desktop reading)

\item {} 
\href{http://gradhandbook.cs.luc.edu/epub/LoyolaComputerScienceGradHandbook.epub}{ePub} (for e-reading devices, e.g. e-readers, tablets, or Adobe Digital Editions on desktop)

\item {} 
\href{http://gradhandbook.cs.luc.edu/}{HTML} (what you are reading now)

\end{itemize}


\chapter{General Information}
\label{general:general-information}\label{general::doc}

\section{Department Office and Personnel}
\label{general:department-office-and-personnel}
The following folks are here to help you.


\begin{threeparttable}
\capstart\caption{Computer Science Department Personnel}

\begin{tabulary}{\linewidth}{|L|L|L|}
\hline
\textbf{
Person
} & \textbf{
Role
} & \textbf{
Contact e-mail
}\\\hline

George K. Thiruvathukal
 & 
Graduate Program Director
 & 
\href{mailto:gpd@cs.luc.edu}{gpd@cs.luc.edu}
\\\hline

Konstantin Läufer
 & 
Department Chair
 & 
\href{mailto:chair@cs.luc.edu}{chair@cs.luc.edu}
\\\hline

Cecie Murphy
 & 
Graduate Program Secretary
 & 
\href{mailto:gpd@cs.luc.edu}{gpd@cs.luc.edu}
\\\hline

Miao Ye
 & 
Computer Systems Manager
 & 
\href{mailto:my@cs.luc.edu}{my@cs.luc.edu}
\\\hline

Jeanmarie Rom
 & 
Department Secretary
 & 
\href{mailto:jrom1@luc.edu}{jrom1@luc.edu}
\\\hline
\end{tabulary}

\end{threeparttable}



\section{Graduate School Offices}
\label{general:graduate-school-offices}
The main office of the Graduate School is on the fourth floor of the Granada Center on the Lake Shore Campus. This office handles admissions and financial aid, and houses the permanent files of all students throughout their graduate careers. The phone number for the Graduate School is (773) 508-3396.


\begin{threeparttable}
\capstart\caption{Graduate School Personnel}

\begin{tabulary}{\linewidth}{|L|L|L|}
\hline
\textbf{
Person
} & \textbf{
Role
} & \textbf{
Contact e-mail
}\\\hline

Jessica Horowitz, PhD
 & 
Assistant Dean
 & 
\href{mailto:jhorow@luc.edu}{jhorow@luc.edu}
\\\hline

Patricia Mooney Melvin, PhD
 & 
Associate Dean
 & 
\href{mailto:pmooney@luc.edu}{pmooney@luc.edu}
\\\hline

Samual Attoh, PhD
 & 
Dean
 & 
\href{mailto:sattoh@luc.edu}{sattoh@luc.edu}
\\\hline
\end{tabulary}

\end{threeparttable}


Dr. Patricia Mooney-Melvin, Associate Dean, and Jessica Horowitz, Assistant Dean, are your primary contacts for inquiries. We recommend that you contact the department faculty and
staff first unless your matter absolutely requires Graduate School assistance.


\section{Director of Graduate Programs}
\label{general:director-of-graduate-programs}
Regardless of the MS program in which you're enrolled, your main adviser will be the Director of Graduate Programs (abbreviated GPD). The current director is Dr. George K. Thiruvathukal. You are expected to confer with him regularly about your course of study in pursuit of your degree. You are encouraged to meet with the Director at other times as well to discuss your progress in the program and your future plans.

Beginning in Fall 2013, we will be assigning a department advisor to you.


\section{Committee on Graduate Programs}
\label{general:committee-on-graduate-programs}
The Director is advised on all matters of policy, admissions, and student status by the Committee on Graduate Programs.


\begin{threeparttable}
\capstart\caption{Graduate Advisory Committee}

\begin{tabulary}{\linewidth}{|L|L|L|}
\hline
\textbf{
Person
} & \textbf{
Role
} & \textbf{
Page
}\\\hline

George K. Thiruvathukal
 & 
Professor, Graduate Program Director
 & 
\href{http://gkt.cs.luc.edu/}{http://gkt.cs.luc.edu/}
\\\hline

Peter L. Dordal
 & 
Associate Professor, Past Graduate Program Director
 & 
\href{http://pld.cs.luc.edu/}{http://pld.cs.luc.edu/}
\\\hline

William L. Honig
 & 
Associate Professor
 & 
\href{http://people.cs.luc.edu/wlhonig}{http://people.cs.luc.edu/wlhonig}
\\\hline

Cecie Murphy
 & 
Graduate Program Secretary
 & 
\href{http://www.luc.edu/cs/}{http://www.luc.edu/cs/}
\\\hline
\end{tabulary}

\end{threeparttable}


A student member will be added.

The faculty members of the Committee also serve as jury for various departmental awards.


\section{Student ID Cards}
\label{general:student-id-cards}
Student ID cards, giving access to library borrowing and other privileges, are available from the Campus Card Office, Sullivan Center, Room 117.


\section{Computer Services}
\label{general:computer-services}
The University maintains several Computer Centers which are available for your use in Sullivan Center, Information Commons, and at 25 E. Pearson. There are also personal computers available for the use of Graduate Assistants in the Graduate Student office space in Crown Center 418.

An account on the student email system is created automatically for each new student. Both the Director of Graduate Programs and the Graduate School will use this account to communicate with you. It is therefore crucial that you check your Loyola e-mailbox on a regular basis. Having your e-mail forwarded to another account can be risky. Some students find the mail doesn't always get delivered. If you find that you are not receiving regular and frequent communiqués from the Department, please notify the Secretary.

Students are encouraged to consult the Systems Handbook for more information about
departmental and university computing resources. See \href{http://syshandbook.cs.luc.edu/}{http://syshandbook.cs.luc.edu/}.


\section{Bulletin Boards and Key Web Resources}
\label{general:bulletin-boards-and-key-web-resources}
The bulletin board outside the CS department on the 5th floor of Water Tower Campus features information regarding the graduate programs—e.g., calls for papers, job offerings, fellowship opportunities. The board in the English main office has departmental notices.

You are also encouraged to keep abreast of the following departmental web resources:


\begin{threeparttable}
\capstart\caption{Key Department Web Sites}

\begin{tabulary}{\linewidth}{|L|L|}
\hline
\textbf{
URL
} & \textbf{
Description
}\\\hline

\href{http://www.cs.luc.edu}{http://www.cs.luc.edu}
 & 
Main Web Site for the CS Department
\\\hline

\href{http://gradhandbook.cs.luc.edu}{http://gradhandbook.cs.luc.edu}
 & 
This handbook's permanent location
\\\hline

\href{http://systems.cs.luc.edu}{http://systems.cs.luc.edu}
 & 
Computer Systems Handbook covering labs, servers, and other computing needs
\\\hline

\href{http://jobs.cs.luc.edu}{http://jobs.cs.luc.edu}
 & 
Informal job postings
\\\hline

\href{http://blog.cs.luc.edu}{http://blog.cs.luc.edu}
 & 
CS Department Blog
\\\hline
\end{tabulary}

\end{threeparttable}



\section{The Emerging Technologies Laboratory}
\label{general:the-emerging-technologies-laboratory}
All graduate students in the BS and MS programs have access to a common space in the Water Tower Campus, Lewis Towers 409, also known as the Emerging Technologies Laboratory. Access is via the Loyola campus card.

Please contact the Department Secretary or Computer Systems Manager (add ref to personnel) for access, if you have trouble entering with your Loyola campus card.


\section{Teaching Opportunities}
\label{general:teaching-opportunities}
In addition to occasional teaching assignments for Graduate Assistants (e.g. to help their instructor when he/she needs to be absent), a number of teaching opportunities are available to experienced graduate students and graduates who have gained experience since leaving our department (with a preference for the latter). The University requires that you have the MS degree; the Department requires that you have taken a wide range of challenging courses, especially in foundational areas such as algorithms, languages, systems, and software engineering with a solid record of achievement in all.

The department chair, in consultation with the Director of Graduate Programs, assigns all classes. Funded students will be assigned classes routinely; unfunded students should apply directly to the department chair. The chairperson will invite applications for a limited number of summer teaching opportunities. The criteria used to assign summer classes to graduate students include: experience and proven success in the classroom; good progress toward the degree; preparedness to teach the courses available; and previous summer teaching (in an effort to distribute summer courses fairly). Contact the department chairperson for further information.


\section{Summer Sessions}
\label{general:summer-sessions}
Two six-week summer sessions are offered through the Department each year, running from May to August. Two to four graduate-level courses are generally offered over the two summer terms.

We encourage research-minded students to consider independent study with a department faculty member.


\section{Housing}
\label{general:housing}
Most graduate students choose to find their own off-campus apartments using the ads in the Reader, the Tribune, and other publicly available sources. Loyola’s Department of Residence Life also offers single- and double-occupancy apartments for graduate students on the Lakeshore Campus and the Water Tower Campus. For additional information, please see \href{http://www.luc.edu/reslife}{http://www.luc.edu/reslife}.


\section{Transportation}
\label{general:transportation}
If you plan to commute to Loyola, there are several parking lots that you may use. The main parking structure, adjacent to Sheridan Road and to the Halas Sports Center, houses the Parking Office, where you may purchase a sticker for annual parking. The fee for one-day parking on campus is \$7.00. At peak class times, available parking can sometimes be scarce. Street parking in the community immediately surrounding campus is scarce. It is also restricted to residents during certain hours, so be sure to read the signs carefully to avoid being ticketed by the police.

Parking is also available near the Water Tower Campus, although it is more expensive. You can have your parking stub stamped at the information desk at the 25 E. Pearson building to receive a modest discount. If you teach or take classes at the Water Tower Campus, you will probably want to acquaint yourself with the University’s inter-campus shuttle bus service. Public buses and the “El” run frequently between campuses and to other points in the city.

Further information on parking is available at \href{http://www.luc.edu/parking/}{http://www.luc.edu/parking/}.
For information on the shuttle, see \href{http://www.luc.edu/transportation/shuttlebus.shtml}{http://www.luc.edu/transportation/shuttlebus.shtml}.


\section{Publication}
\label{general:publication}
Research-oriented graduate students (especially those pursuing the \emph{thesis option} in MS CS) are encouraged to pursue the publication of one or more journal articles during their graduate careers. In the current job market, publication is an important means of demonstrating to prospective employers a high level of motivation and professional competence. It can also be helpful for your future pursuits as a doctoral student (elsewhere).

Three ways to prepare for this goal are 1) to read journals in your fields of interest regularly in order to become familiar with both current scholarship and the requirements of scholarly publication; 2) to review the MLA Directory of Periodicals in order to learn what different journals expect or demand; and 3) to approach seminar papers, especially those in your field(s) of interest, as potential publications, possibly even as publications targeted to a particular journal. In developing a paper for publication, students are of course well advised to work closely with their seminar instructors or faculty mentors. In the recent past the Director of Graduate Programs and members of the graduate faculty have offered semester-long writing workshops for students writing for publication or writing their first conference presentations. You are advised to take advantage of such opportunities.


\section{Conference Presentations and Travel Funding}
\label{general:conference-presentations-and-travel-funding}
Presentation of conference papers is an important part of students’ professional development, and PhD students in particular should aim to give at least two papers during their graduate careers—preferably including professional and not just graduate-student conferences.

Each semester the Graduate School has funds to support graduate student travel for the purpose of presenting papers or chairing sessions at conferences. Since funds are limited, students should apply immediately upon acceptance of their papers or sessions. (Forms are available at \href{http://www.luc.edu/gradschool/servicesandresources\_forms.shtml}{http://www.luc.edu/gradschool/servicesandresources\_forms.shtml}).  The Department supplements these funds when the Graduate School’s funds are exhausted or when a student is presenting a paper at a second conference within one academic year. (Department forms are available on the Graduate Programs website \href{http://luc.edu/english/links.shtml}{http://luc.edu/english/links.shtml}.) The Department also has funding for attending summer seminars or conducting dissertation research at a research library outside Chicago. These are competitive awards offered each semester. The current subvention from the Graduate School for travel is \$400. Departmental travel support, which is funded primarily from the EGSA Student Activities budget, is likely to provide a much lower level of sponsorship. Research awards are provided through the Department’s Gravett-Tuma fund and are around \$500 (depending on the number of awards and the funds available).


\section{Career Center}
\label{general:career-center}
The University’s Career Center, where each student entering the job market should establish a dossier containing letters of recommendation, is located in Sullivan Center, Room 295 (508-7716), with a very helpful website: \href{http://www.luc.edu/career/RamberLink\_Login.html}{http://www.luc.edu/career/RamberLink\_Login.html}

Students are encouraged to check our information jobs listings as well at \href{http://jobs.cs.luc.edu}{http://jobs.cs.luc.edu}. We are routinely contacted by employers who are seeking interns, consultants, and ``permanent'' employees.


\section{Department Awards}
\label{general:department-awards}
Each year the Department recognizes exceptional graduate students in all of our degree programs. Awards are given for academics and service. The Graduate Advisory Committee is responsible for selecting the award winners.


\chapter{Regulations and Procedures}
\label{regulations:regulations-and-procedures}\label{regulations::doc}\label{regulations:index-0}

\section{Course Loads}
\label{regulations:course-loads}
A full-time student will usually carry three courses per semester. Course loads for part-time students are worked out on an individual basis. All students funded by the Department or the Graduate School are considered full time.

We do not encourage students to take more than 3 courses a term. Please contact the Graduate Program Director if you have a need to increase your load beyond what is customary. We encourage you to consider taking summer courses if you need to complete your degree more quickly.


\section{Registration}
\label{regulations:registration}
Both new and continuing students must complete the process of registration before every semester in which they are either attending classes or writing their dissertations. A schedule of courses for the upcoming term is available a few months before classes are scheduled to begin. In order to register for graduate courses you must first consult with the Director of Graduate Programs. Once your selections are approved, they will be entered into the system by the Graduate Programs Secretary, completing the registration process. It is your responsibility to check LOCUS to verify your registration each semester. Students must maintain continual registration throughout their years in the program or risk having to apply for reinstatement and pay both a penalty and back fees.

After registration opens you can register yourself through LOCUS for
most courses with just your student ID number.  New students should get
course advising first from the department, and all others are welcomed
to get advising.  You cannot register yourself for Comp 490 or 499.  See
the sections below concerning independent study and internships.  Other
courses that can have complications registering and dropping are the
CSIS courses, depending on when you want to register or drop a course.
See the section on CSIS Courses below.

It is in your best interest to register early, to get into sections you
want and have ample time to detect and clear up any possible
registration block that may have been placed.  Common examples are an
immunization block or a bursar block.  Students have been given late
fees or had a lot of trouble due to delay from such blocks.  Deal with
these and clear them out early.  One misleading feature in LOCUS is that
all MS students have a Dean's time limit block, but it does not activate
until you have been an MS student for five years - do not worry about
that one.


\section{Independent Study}
\label{regulations:independent-study}
If you are doing an independent project for Comp 490, you need to find a
faculty member to supervise your project and have the faculty supervisor
email let the department to register you.  This does not generally make
sense in your first semester.  It helps if you and the faculty member
are familiar with each other.

The typical approach is for the student to confirm the details of the
agreement in an email to the faculty supervisor, being sure to include
the exact course, semester and units your have agreed on.  The
supervisor then forwards the email with his or her approval to the
department administration.  These courses are for 1-6 units.  See the
section below on variable hour courses.

A good feature of Independent Study is that you can have your transcript
show a course title that is specific to your particular project.  Fill
out
\href{http://www.luc.edu/gradschool/forms/requestfortitle.pdf}{http://www.luc.edu/gradschool/forms/requestfortitle.pdf}and
get the signature of your supervisor and the gpd.


\section{Internships}
\label{regulations:internships}
First, you need to find a job.  There are online, searchable listings
through the university career center,
\href{http://www.luc.edu/career}{http://www.luc.edu/career}.  Some of these
listings directed specifically at CS students are duplicated on paper in
big folders in the department administrative offices.  When you find a
job, contact the GPD and get your job description approved.  Communicate
the number of units desired and the semester in an email so we have a
record, and we can register you.  Internships are for 1-6 units.  See
the section below  on variable hour courses.

Separately download, print, and fill out the MOU form linked on the 499
web page.  You will also need your job supervisor's signature.  Get the
finished form to the GPD.  The MOU can be turned in just after you start
your job. It can be scanned and emailed or turn in paper to the
department staff.
There are three related but different terms:  \textbf{job}, \textbf{internship},
and \textbf{CPT} (for F-1 visa students - see below).  You can have a job and
not have it be an internship for academic credit, or you can have a job
that goes on longer than an internship.  Also if you are doing an
academic internship, your job employer does not need to classify your
position as ``internship''. If you are doing an academic internship, the
MOU indicates only two small requirements for your employer during your
time in the academic internship:  The bulk of your duties must be
related to doing computer science in the real world, and the supervisor
will write a few line email a the end of the time for the academic
internship indicating your successful completion of all the hours
required for the academic internship.


\section{Curricular Practical Training (CPT) (For F-1 students)}
\label{regulations:curricular-practical-training-cpt-for-f-1-students}
You should consult with the \href{http://www.luc.edu/oip}{International
Center} for the full legal details of CPT.
 Here are a few of the important points.  If you get a job on campus,
like the considerable number of students who have worked for Loyola's
Information Services, you do not need to be doing CPT and no CPT
restrictions apply.  If you want to work off campus for pay, then you
need to be doing CPT, and there are a number of requirements.

First of all, you need to have been a full-time F-1 student anywhere in
the United States for two semesters.  If you want to count a summer (as
students starting in Spring or Summer are likely to want), there is a
confusing point:  The number of units needed to be full-time to count as
leading up to CPT are different than the number of units a student
starting in summer needs to be in full-time status!  Be sure to check
with the International Office for the exact current details.

The procedure to start CPT:
\begin{enumerate}
\item {} 
You must submit a CPT application with the Graduate Secretary and
include a written job offer (which can be an email) from the
prospective employer.   The form is recently advised and may still
not be on the OIP webiste.  If their form
\href{http://www.luc.edu/oip/pdfs/verification.pdf}{http://www.luc.edu/oip/pdfs/verification.pdf},
includes an option for an instructor signature, then it is the most
recent version of the updated form.  If not, use our local copy at
Revised CPT Form.  The
CPT application must be delivered to the International Office, and
they will issue the work permit.

\item {} 
The CPT can either be tied to a Comp 499 course OR for no extra
tuition you are likely to be able to tie it to a course you are
already planning to take:  There is a form on the International
Office site for getting the approval of an instructor to pair the
internship with a course in the same semester, or for a fall course
and an internship in the previous summer.  The form indicates you
also need the GPD's approval.  In you do the no-extra-tuition option,
you are able to work but you get no further credit toward graduation,
and the course instructor is responsible for determining any extra
reporting or work you do to tie the internship to the instructor's
course.

\end{enumerate}


\section{Variable Hour Courses}
\label{regulations:variable-hour-courses}
Comp 490 and 499 are for 1-6 units.  Up to 6 units total can be counted
toward graduation, counting all the times you register for these two
courses.  In practice that means 3 or 6 units since all other courses
are 3 units.  You do not need to take a multiple of three units at a
time.  What matters is the total when it is time to graduate.
International students on F-1 visas:  This is particularly useful for
you, who need to be registered for off-campus internships and who need 8
units, not 9, to be considered full time in fall and spring.  Examples:
 You can do a 1-credit internship/CPT in the summer and do a 2-credit
independent study in another semester.  If you want to extend your work
time at the end of your studies and would normally graduate in the
spring, you could do two 2-credit independent studies earlier, leaving 2
credits needed in the final summer and do a 1 credit CPT in first summer
session, so you can start summer work as soon as possible in the summer,
and do another 1 credit CPT in the second summer session, delaying
completion, so post-graduation OPT does not need to start until after
the second summer session.  Some F-1 students also do a unit of CPT
beyond the 30 credits needed for graduation, so they are eligible to
work.


\section{CSIS Courses}
\label{regulations:csis-courses}
CSIS courses are special sections set up to give you credit in the
Computer Science MS program for courses offered by the Graduate School
of Business.  They broaden the Computer Science offerings and let you
essentially take GSB courses at the Graduate School's much lower tuition
rate.  There are a number of special considerations coming from the fact
that GSB courses are quarter courses.  They have the same holidays as in
The Graduate School semester system, but exam times or term start times
or both are different.   The main administrative issue is that this
confuses LOCUS, the school online administration system.  Fall Quarter
starts with Fall semester, but ends in November.  Winter Quarter goes
from November into February, spanning parts of both Fall and Spring
Semesters.  A fairly arbitrary decision was made to list Winter Quarter
CSIS courses under Fall semester.  Spring Quarter Courses do not start
until February.

Because Winter Quarter spans two semesters, it is very important to look
at the Spring Semester schedule for COMP courses before registering for
a Winter Quarter course.  Registration for a Winter Quarter course will
make it impossible for you to register for a Spring Semester COMP course
on the same night.

LOCUS lists CSIS courses as semester courses, so if you look at your
current course list in October, you will see both Fall and Winter
Quarter courses included!  \emph{You} have to know the \emph{real}calendar.
Particular issues arise with registering for and dropping CSIS courses
outside the times LOCUS is expecting.  If you register and add or drop
in the regular semester time limits (much earlier than the time Winter
and Spring quarter courses actually start), then you should be able to
do your registration changes by yourself, online, in LOCUS, with no
problem.  Please do that where possible.  On the other hand, \textbf{if you
want to make changes closer to the time Winter and Spring Quarter
courses actually start}, you should make all registraion requests
through the GPD, \href{mailto:gpd@cs.luc.edu}{gpd@cs.luc.edu}.  Since the department ends up making
registration changes which are very important to you, we need explicit
directions and you need to indicate clear knowledge of the ramifications
of your choices.  Include the following in your email:

\textbf{Registration request} to \href{mailto:gpd@cs.luc.edu}{gpd@cs.luc.edu} after the normal LOCUS
semester registration time limit and before the GSB registeration time
limit for Winter or Spring:
\begin{enumerate}
\item {} 
Include a direct request like ``Please register me for CSIS XXX
Section YYY for ZZZ Quarter.'' \emph{not} an indirect question like ``Would
it be OK if I register for....?''

\item {} 
Include your full name and Student ID number.

\item {} 
Explicitly acknowledge the drop deadlines and the timeframe and
manner you must notify us to get you dropped (as further discussed
below).   You could include something like ``I know I must email you
with an explicit request to drop the course by XX/XX/XXXX if I want
no trace left fo the course and by YY/YY/YYYY to avoid tuition, but
still get a W on my transcript.''  The dates are publicized at the
\href{http://www.luc.edu/gsb/academics\_calendars.shtml}{GSB web site}.
We will try to include them also on our course web pages for Winter
and Spring Quarters.

\end{enumerate}

\textbf{Drop requests} after the normal LOCUS semester drop/add time limit,
but inside the limits set by the Graduate School of Business:
\begin{enumerate}
\item {} 
Within the limits set by the GSB, make the drop in Locus.  This will
drop you and note the time of your decision.

\item {} 
Email \href{mailto:gpd@cs.luc.edu}{gpd@cs.luc.edu} and explain when you dropped what exact course,
and ask us to backdate the withdrawal to make up for LOCUS's
incorrect understanding of dates.

\item {} 
Include your full name and Student ID number.

\end{enumerate}

The time of dropping the course is crucial in determining its effect.
Be aware of the GSB deadlines for getting the course dropped with no
trace and the later deadline for avoiding  tuition.  We will be correct
things if you are before the GSB deadlines.  See below under Dropping a
Course for further discussion of the categories.


\section{Graduation}
\label{regulations:graduation}
Degrees are conferred in May, August, and December.  You must do
paperwork \textbf{*way in advance*} of graduation or the official conferral
of your degree will be \textbf{*postponed*}.  I will not be able to appeal
this for you.  Note that there are only graduation \emph{ceremonies}in May.

\textbf{Deadlines}:   December 1 for Spring/Summer graduation, August 1 for
Fall graduation.  See the discussion of ceremonies below if you want to
participate in a graduation ceremony and you graduate in Summer or Fall.

\textbf{Procedure}:
\begin{enumerate}
\item {} 
Go into Locus and submit your application for graduation, making sure
you end up with a paper copy of the completed form.

\item {} 
Get the form and \$75 to the Bursar's office, and  get the form
stamped.

\item {} 
Deliver the stamped form to the Graduate School on the fourth floor
of Granada Center (not the CS dept.).  You can do this in person, by
mail, or by emailing a scanned copy.  The Graduate School should
confirm their receipt of the form.

\end{enumerate}

If your last course is a CSIS course in Winter Quarter, still register
for Fall graduation, since Winter Quarter courses are listed under Fall
semester in LOCUS.  Of course you will not really graduate until after
Winter Quarter courses end in February.

\textbf{Graduation Ceremonies only in May}:  If you have only one course left
for summer, you can ask to participate in the \emph{previous}May
graduation.  To do you must apply by the deadline listed above and
promptly email the GPD, asking for approval to walk in the May
ceremony.  If you graduate in the Summer or Fall, you can choose to
return to participate in the \emph{following}May graduation ceremony
(unless you already participated in the previous May graduation, as
discussed above).


\section{Leave of Absence}
\label{regulations:leave-of-absence}
Once you start grad school, the default assumption is that you will be
enrolled each fall and spring until you sign up for graduation and
graduate.  If you need to interrupt your studies before that, we ask
that you apply for a leave of
absence \href{http://www.luc.edu/gradschool/forms/leaveofabsence.pdf}{http://www.luc.edu/gradschool/forms/leaveofabsence.pdf}.
 There is no direct penalty for forgetting notification, but it helps
the department to know what is doing on.

  Whether or not you file a Leave of Absence form, you will still need
to file a Request for Reinstatement form,
\href{http://www.luc.edu/gradschool/forms/REINSTATEMENT\_REQUES.doc}{http://www.luc.edu/gradschool/forms/REINSTATEMENT\_REQUES.doc}

,  on your return, before you can register again. Turn in this form
early enough to register promptly!


\section{Dropping a Course, Avoiding Extra Bills}
\label{regulations:dropping-a-course-avoiding-extra-bills}
You should always be able to withdraw yourself from the course in LOCUS,
no matter how you got registered for a course: by yourself in LOCUS, by
a request to the department staff, or off of a waiting list, .  If you
are sure you want to withdraw from a course, do not waste time emailing
the department for help, just do it yourself.  The date that the
withdrawal is entered into LOCUS affects whether you get a W on your
transcript, and whether tuition is still due.  Different dates apply.
 Be sure to look at the Academic Calendar for the given semester.  Once
you are registered, merely not attending class does \textbf{NOT} extend these
dates.
\begin{itemize}
\item {} 
Withdrawal with no trace:  Generally by the end of the first week of
Fall and Spring semesters.  Generally only through the first Tuesday
of the semester for Summer session.

\item {} 
Withdrawal with only a W on the transcript, and no tuition due:
 Generally during the second week of Fall and Spring semesters.
 Sometime during the first week in summer sessions.  Be sure to check
the Academic Calendar.  A W has no academic consequences.  It is just
a historical record of you changing your mind.

\item {} 
Withdrawal later during classes:  W on the transcript and a partial
or complete tuition penalty.  Do not get yourself into this situation
just by not paying attention!

\end{itemize}

The categories are the same for CSIS courses, but the procedures can be
more complicated.  See the section on CSIS Courses above.


\section{Changing your chosen MS Program}
\label{regulations:changing-your-chosen-ms-program}
It is easy to switch between our MS degree programs in the department.
Submit the form
\href{http://www.luc.edu/gradschool/pdfs/changeofstatus.pdf}{http://www.luc.edu/gradschool/pdfs/changeofstatus.pdf}
to the GPD, who will review it and forward it to the Grad School.


\section{Transfer of Credit into the Loyola MS Program from Earlier Graduate Work}
\label{regulations:transfer-of-credit-into-the-loyola-ms-program-from-earlier-graduate-work}
During your first semester, you can apply to transfer up to 6 units of
previous graduate work relevant to your current program.   Your official
transcripts need to show B or better in relevant courses.  (In
particular, we must have your official transcripts already!)
International students, read the section below for further requirements.
 Submit the form
\href{http://www.luc.edu/gradschool/forms/transfer\_credit.pdf}{http://www.luc.edu/gradschool/forms/transfer\_credit.pdf}
to the GPD, after you have submitted the necessary transcripts.
 Although official transcripts are needed to forward the request to the
Grad School for final approval, you are welcomed to show unofficial
transcripts to the GPD to see if you have appropriate courses.


\section{Further International Transcript Credit Transfer Requirements}
\label{regulations:further-international-transcript-credit-transfer-requirements}
International transcripts need only a \emph{general} evaluation by ECE or
WES for \emph{admission}, but they need a \emph{course by course} evaluation to
\emph{transfer} international graduate credit.  It is most economical to ask
for the course by course evaluation the first time transcripts are
submitted to ECE or WES if you are expecting to get transfer credit.


\section{Grades}
\label{regulations:grades}
The grading system used in the Graduate School is as follows:


\begin{threeparttable}
\capstart\caption{Grading System}

\begin{tabulary}{\linewidth}{|L|L|}
\hline
\textbf{
Grade
} & \textbf{
Grade Points
}\\\hline

A
 & 
4.00
\\\hline

A–
 & 
3.67
\\\hline

B+
 & 
3.33
\\\hline

B
 & 
3.00
\\\hline

B–
 & 
2.67
\\\hline

C+
 & 
2.33
\\\hline

C
 & 
2.00
\\\hline
\end{tabulary}

\end{threeparttable}



\begin{threeparttable}
\capstart\caption{Other Grading Codes}

\begin{tabulary}{\linewidth}{|L|L|}
\hline
\textbf{
Grade
} & \textbf{
Explained
}\\\hline

I
 & 
Incomplete
\\\hline

W
 & 
Withdrawal
\\\hline

WF
 & 
Withdrawal, Failure
\\\hline

CR
 & 
Credit
\\\hline

NC
 & 
No Credit
\\\hline

AU
 & 
Audit
\\\hline
\end{tabulary}

\end{threeparttable}


For further information on Loyola’s grading policy, consult the Graduate School Catalog.

Graduate students in the English Department are expected to maintain an average of not less than B (3.0). Those who fail to meet this requirement may be dismissed. No more than two grades below B and no grades of D or F may be counted as fulfilling degree requirements. PhD students who find themselves receiving primarily B’s (or below) should consider seriously whether pursuing a doctorate in English remains the most gainful use of their time.


\section{Withdrawal}
\label{regulations:withdrawal}
Withdrawal from graduate courses is uncommon; however, students who do withdraw from a course must first consult with the Director of Graduate Programs before dropping through LOCUS. Students should check the academic calendar for deadlines on withdrawing for full or partial refunds, and with a W or F grade. Students are responsible for withdrawing themselves from classes.


\section{Incompletes}
\label{regulations:incompletes}
Faculty may assign the grade of I to a student who has not completed the assigned work by the end of the term. This grade is not assigned automatically; rather, it is up to the student to work out with the instructor a plan, including a deadline, for completing the work for the course.

The Director of Graduate Programs must sign off on incompletes before they will be awarded.Students need to download the change-of-grade form from the Graduate School website and give it to the GPD when the incomplete is approved.

Under the Graduate School regulations, a student has one semester to complete the course. If the student does not turn in the work by the deadline, the I will automatically become an F.  Please read the new policy on the Graduate School web page at \href{http://www.luc.edu/gradschool/academics\_policies.shtml\#grades1}{http://www.luc.edu/gradschool/academics\_policies.shtml\#grades1}.

Although it is not uncommon for graduate students to take an occasional Incomplete, it is of course better not to take an incomplete when possible. Making up an incomplete course often proves harder than students expect, particularly if much time has elapsed since the end of the course. In any case, faculty members have various policies regarding Incompletes, so it is advisable to discuss the matter with your instructor as early as possible if you anticipate the need for an Incomplete. You must also complete a form (available on our Web site at \href{http://}{http://} www.luc.edu/english/links.shtml) and have it signed by the GPD. No more than one I can be requested in a semester, unless there are extenuating circumstances (e.g., a serious illness).


\section{Leaves of Absence}
\label{regulations:leaves-of-absence}
Official leaves of absence are intended for students who wish to discontinue temporarily their graduate studies due to special circumstances (e.g., medical, personal, or professional reasons). A leave of absence postpones all deadlines concerning completion of degree requirements for the duration of the leave. A student requesting a leave must complete a Leave of Absence form and contact the Director of Graduate Programs, who then makes a recommendation on the student’s behalf to the Graduate School. Decisions regarding the approval of leaves of absence rest with the Graduate School.

Leaves of absence may be requested for a semester or for a full academic year. In order to be reinstated to active status, the student must notify the Graduate School in writing upon returning from a leave. Unless the student is granted a renewal of a leave, he or she must return to active status in the semester following its expiration. Failure to do so may result in withdrawal from the program.

The Leave of Absence form is found in the GPRS system.


\section{Advanced Standing/Transfer Credits}
\label{regulations:advanced-standing-transfer-credits}
It is ordinarily expected that all work for the Master’s degree will be completed at Loyola. Upon the recommendation of the Director of Graduate Programs, however, and with the approval of the Dean, up to six hours of credit for graduate work at another university may be counted toward the degree.


\section{Academic Honesty}
\label{regulations:academic-honesty}
Although academic dishonesty can take many forms, in our field it manifests primarily as plagiarism of text our source code. The Graduate School Catalog defines plagiarism as “the appropriation for gain of ideas, language or work of another without sufficient public acknowledgement that the material is not one’s own.” As a graduate student, you very likely have a good understanding of the boundaries of what is acceptable and what is not. If you are ever uncertain, it is of course best to consult the Director of Graduate Programs or another faculty member.

The penalty for an instance of plagiarism is, at a minimum, failure on the assignment, which may well be tantamount to failure in the course. A serious breach or a pattern of dishonesty can lead to expulsion from Loyola. Although quite rare in our department, cases have occurred in the past and have resulted in dismissal.


\section{Grievance Procedure}
\label{regulations:grievance-procedure}
Students, faculty, and administrators are strongly encouraged to resolve any problems they encounter in the academic process through informal discussion. If you are unable to resolve a problem with a member of the staff or faculty, or if you wish to lodge a formal complaint, you should first meet to discuss the matter with the Director of Graduate Programs. If the problem cannot be satisfactorily resolved by the GPD, it will be taken up by the Department Chair. Violations of the University’s ethical standards not resolvable within the Department may call for the use of the Graduate School’s grievance procedure. Students wishing to initiate a grievance must do so in writing to the Dean. Further information can be obtained from the Graduate School office.


\chapter{Master of Science Programs}
\label{ms_programs::doc}\label{ms_programs:master-of-science-programs}

\section{MS CS}
\label{ms_programs:ms-cs}

\section{MS SE}
\label{ms_programs:ms-se}

\section{MS IT}
\label{ms_programs:ms-it}
All of this needs to be written for CS.


\section{MS Thesis Option}
\label{ms_programs:ms-thesis-option}
A thesis is not required for the MA degree. Students may choose to write a thesis in lieu of taking two classes (6 credits), although course work is strongly recommended over the thesis option. Students wishing to do a thesis should discuss this option as early as possible with the Director of Graduate Programs. Regarding thesis procedures, see the Graduate School Catalog.


\section{Time and Residence Requirements}
\label{ms_programs:time-and-residence-requirements}
Normally it takes one year of full-time study to earn a master’s degree in English from Loyola. Students who attend part time must complete the program within five years. This period may be extended only by special action of the Dean of the Graduate School.


\section{Application for Degree}
\label{ms_programs:application-for-degree}
It is the exclusive responsibility of all graduate students to inform the Graduate School office of their intention to graduate with an MA degree. Once you have determined when you will receive your degree (e.g., August, December or May), you must apply for graduation through LOCUS. The Graduate School sets the deadlines (typically December 1 for a May degree, February 1 for an August degree, and August 1 for a December degree). Check the academic calendar for the Graduate School available through the main page of the Loyola website.

After completing the summer courses and passing the exams, you'll be certified for an August degree. Because we have only one ceremony per year when students ``walk,'' however, August degree students walk in the May ceremony the following year.


\chapter{Assistantships}
\label{assistantships::doc}\label{assistantships:assistantships}

\section{Teaching Assistantships}
\label{assistantships:teaching-assistantships}
Each year the Department of English offers a number of Graduate Assistantships for new and continuing students. Unlike many other universities, however, Loyola concentrates on developing assistants’ skills before allowing them to assume sole responsibility for a composition course. Rather than being immediately assigned to teach a class during their first semester of graduate school, assistants first gain experience tutoring undergraduates in the Writing Center. First semester they tutor for two shifts. During their second semester, assistants work with a faculty member in a Mentorship Program that allows them to assume some grading and instructional duties within the mentor’s composition class, and tutor for one shift.

In the second year (or first year post-MA), assistants teach one composition course each semester, receiving informal support and guidance from the mentor of the previous year. In the third, fourth and fifth years, assistants normally work as teaching assistants in a literature course, teach one literature course each per term, or serve as research assistants for one semester. In addition, one student each year is assigned as the assistant director of the Writing Center. Graduate instructors never teach more than one class per term.

Policy on “floaters”

All third, fourth, and fifth-year graduate assistants who are not serving as instructors-of-record also serve as “floaters.” Floaters take courses for faculty who must miss a class because of illness, jury duty, professional conferences, or other University-sanctioned events. Students serving as floaters in a given semester cannot turn down a request to take a faculty member’s class unless they are in class themselves. In turn, the department makes every effort to distribute the workload equitably among the floaters in any one semester.

Because floaters are not necessarily trained in the content area of the faculty member’s course, faculty members devise assignments with this in mind. Floaters may show a film, conduct a writing workshop, hand back and discuss papers, or teach a poem, essay or story that they already know or can prepare with minimum effort and much advanced notice. The faculty member should provide detailed notes for what he/she expects to be covered in class that day.

Graduate Assistantships are renewable for a total of five years of support, if the students remain in good standing and are making good progress toward the degree. After that, students can apply through the Department for a range of fellowship opportunities made available by the Graduate School (see below).


\section{Funding for Advanced Students}
\label{assistantships:funding-for-advanced-students}
Students who have completed course work, passed the PhD exam, and received approval of their dissertation proposals are eligible to apply for the Arthur J. Schmitt Dissertation Fellowships. The award, which is not renewable, requires no service. Advanced students are also eligible to apply for a very limited number of Teaching Scholars Awards, whose service requirements include the teaching of three courses, and Advanced Fellowships with no teaching responsibilities.

For further information on these awards, administered by the Graduate School, please see \href{http://www.luc.edu/gradschool/admission\_financialaid.shtml}{http://www.luc.edu/gradschool/admission\_financialaid.shtml}.


\chapter{Acknowledgments}
\label{acknowledgments::doc}\label{acknowledgments:acknowledgments}
We wish to acknowledge the following for their contributions that made this possible.
\begin{itemize}
\item {} 
We actually started from the general structure of the English department's graduate studnet handbook.

\item {} 
Dr. Harrington (past Graduate Program Director for Computer Science) put together some administrative resources that proved helpful for our department-specific policies/procedures section.

\end{itemize}


\chapter{Indices and tables}
\label{index:indices-and-tables}\begin{itemize}
\item {} 
\emph{genindex}

\item {} 
\emph{modindex}

\item {} 
\emph{search}

\end{itemize}



\renewcommand{\indexname}{Index}
\printindex
\end{document}
